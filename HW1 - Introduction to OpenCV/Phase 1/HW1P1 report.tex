\documentclass[a4paper,12pt]{article}
%\pagestyle{empty}
\usepackage{geometry}
 \geometry{
 a4paper,
 total={170mm,257mm},
 left=20mm,
 top=20mm,
 }
 \usepackage{float}
\usepackage{graphicx}
\usepackage{amssymb}
\usepackage{enumitem}
\usepackage{xepersian}

\usepackage{bbm}
\settextfont{B Nazanin}
\DefaultMathsDigits
\renewcommand{\labelitemi}{$\circ$}


\begin{document}
\begin{center}
به نام خدا
\end{center}
\begin{large}
بینایی ماشین
\hspace{11.5cm}
نیمسال دوم 98-99
\\
گزارش فاز یک تمرین اول\hspace{7.5cm}  دانشجو: مهرسا پوریا (95101247)
\end{large}

\noindent\rule{\textwidth}{1pt}

\begin{center}
\textbf{برای کم شدن حجم آپلود شده عکسهای ورودی و عکسهای خروجی مربوط به سوال 5 در خروجی نیامده اند؛البته عکسهای سوال ۵ در متن گزارش موجودند. اما برای ران کردن کد لازم است عکسها و ویدیوهای ورودی به محل اجرا افزوده شود. برای بستن پنجره ها و رفتن به مرحله ی بعد در هنگام اجرای کد از کلید \lr{e} استفاده کنید.}
\end{center}
\section*{الف)}

\subsection*{سوال اول}
\subsubsection*{الف)}

در این قسمت مطابق خواست سوال تصویر زیر حاصل شده است. برای نوشتن متن از تابع 
\lr{cv.putText}
و برای تبدیل تصویر خوانده شده توسط دستور 
\lr{cv.imread}
که خود به فرمت 
\lr{BGR}
ذخیره شده است به 
\lr{RGB}
و 
\lr{GRAY}
از تابع 
\lr{cv.cvtColor}
استفاده شده است.

\begin{figure}[H]
\center
\includegraphics[scale=0.15]{output_a1a.png}
\caption{از سمت راست \lr{GRAY} و \lr{RGB} و \lr{BGR}}
\end{figure}

\subsubsection*{ب)}
در این قسمت با آزمایش و خطا محل توپ را یافتم و با \lr{cv.rectangle} مستطیل سبز را رسم و در نهایت توپ را در جای دیگری از تصویر ذخیره کردم.

\begin{figure}[H]
\center
\includegraphics[scale=0.3]{output_a1b.png}
\caption{یافتن و کپی توپ به جای دیگر}
\end{figure}

\subsection*{سوال دوم}

در ابتدا برای اینکه در دوران تصویر به درستی نمایش داده شود؛ یک صفحه سیاه مربعی به ضلع قطر بزرگ تصویر ایجاد کرده و تصویر داده شده را در مرکز آن کپی کردم تا بعد دوران مشکلی در نمایش به وجود نیاید. برای ساخت اسلایدر از تابع 
\lr{cv.createTrackbar}
استفاده کردم و برای دوران درجه خوانده شده از اسلایدر را به تابع 
\lr{cv.getRotationMatrix2D}
دادم و با استفاده از ماتریس دوران خروجی آن و تابع 
\lr{cv.warpAffine}
دوران را انجام دادم. مختصات نقطه بعد از دوران هم با استفاده از ضرب ماتریسی مختصات نقطه در ماتریس دوران حاصل شد و دوایر و خط به ترتیب توسط تابعای
\lr{cv.circle}
و
\lr{cv.line}
رسم شده اند. برای نمونه دو حالت دوران در زیر رسم شده است.

\begin{figure}[H]
\center
\includegraphics[scale=0.18]{66d.png}
\caption{دوران 66 درجه}
\end{figure}

\begin{figure}[H]
\center
\includegraphics[scale=0.18]{248d.png}
\caption{دوران 248 درجه}
\end{figure}

\subsection*{سوال پنجم}
\textbf{(در این سوال برای بستن پنجره باز شده و رفتن به بعدی کلید \lr{e} بفشارید)}
در این سوال با توجه به اینکه برای ابعاد پنجره و همچنین نوع کرنل های مختلف میخواستم خروجی را چک کنم از اسلایدر برای گرفتن سایز پنجره و نوع کرنل استفاده کردم که این امر تست را بسیار راحت کرد. 
\\
در هر دو تبدیل 
\lr{Erosion}
و
\lr{Dilation}
از سه نوع کرنل
\lr{RECT}
و
\lr{ELLIPSE}
و
\lr{CROSS}
استفاده شد که به ترتیب مقدار یک در همه المانها؛ ناحیه ای شبیه بیضی و ناحیه ای شبیه + دارند و در بقیه جاها صفر میباشند. برای تولید این کرنلها از تابع 
\lr{cv.getStructuringElement}
استفاده شده است.

\subsubsection*{\lr{Erosion}}
در 
\lr{Erosion}
مقدار یک پیکسل با توجه به شکل کرنل با مینیمم مقادیر تصویر در همسایگانی که مقدار کرنل در آنها یک است جایگذاری میشود(معادله ۱). در واقع با این تبدیل انتظار داریم زمینه تصویر تیره تر شود. در تصویر داده شده که خود نیز تیره است این کار باعث تقویت قسمتهای تیره و کمتر شدن قسمتهای روشن درون قسمت های تیره مثل چشمهای آدمک میشود. آز آنجا که در تصویر داده شده اکثر شکلها دایره ای هستند بهترین نوع کرنل به نظر من کرنل 
\lr{ELLIPSE}
است که به صورت دایره ای جایگذاری را انجام میدهد. برای هر سه نوع کرنل و سایز پنجره 31 خروجی به صورت زیر در می آید. با توجه به اینکه کرنل بیضوی خروجی بهتری دارد در همه قسمت های بعد برای بررسی اثر پنجره از آن استفاده میکنیم. برای 
\lr{Erosion}
با فیکس کردن نوع کرنل در صفحه بعد برای سه اندازه 3 و 7 و 15 خروجی را رسم میکنیم؛ ملاحضه میشود با افزایش اندازه پنجره نقاط تیره بیشتر تقویت و بر خلاف آنها نقاط سفید محصور در سیاه ها مثل چشم تضعیف میشوند.

\begin{equation}
	𝚍𝚜𝚝(x,y)=\min(\prime{x},\prime{y})_{kernel(\prime{x},\prime{y}) \neq 0} 𝚜𝚛𝚌(x+\prime{x},y+\prime{y})
\end{equation}
	
\begin{figure}[H]
  \centering
  \begin{minipage}[b]{0.4\textwidth}
    \includegraphics[width=\textwidth]{rect31.png}
    \caption{\lr{rect kernel, wSize = (31,31)}}
  \end{minipage}
  \hfill
  \begin{minipage}[b]{0.4\textwidth}
    \includegraphics[width=\textwidth]{cross31.png}
    \caption{\lr{cross kernel, wSize = (31,31)}}
  \end{minipage}
\end{figure}

\begin{figure}[H]
\center
\includegraphics[scale=0.2]{ellipse31.png}
\caption{\lr{ellipse kernel, wSize = (31,31)}}
\end{figure}

\begin{figure}[H]
  \centering
  \begin{minipage}[b]{0.4\textwidth}
    \includegraphics[width=\textwidth]{ellipse3.png}
    \caption{\lr{erosion, wSize = (3,3)}}
  \end{minipage}
  \hfill
  \begin{minipage}[b]{0.4\textwidth}
    \includegraphics[width=\textwidth]{ellipse7.png}
    \caption{\lr{erosion, wSize = (7,7)}}
  \end{minipage}
\end{figure}

\begin{figure}[H]
\center
\includegraphics[scale=0.2]{ellipse15.png}
\caption{\lr{erosion, wSize = (15,15)}}
\end{figure}

\subsubsection*{\lr{Dilation}}
بر خلاف تبدیل قبلی این بار هر پیکسل با ماکزیمم همسایگانگش که در کرنل مربوطه مقدار یک دارند جایگذاری میشوند (معادله ۲). بنابراین انتظار داریم در این تبدیل نقاط سیاه تر محصور در ناحیه های روشن تضعیف شوند و نقاط روشن محصور در نقاط سیاه تقویت شوند؛ که این استدلال با خروجی ها هماهنگی دارد. با افزایش سایز پنجره نیز اثرات تبدیل شدید تر میشود به طور خاص در این عکس چشم ها روشن تر و غبارها ناپدیدتر میشود و بدن آدمک محوتر میشود.

\begin{equation}
	𝚍𝚜𝚝(x,y)=\max(\prime{x},\prime{y})_{kernel(\prime{x},\prime{y}) \neq 0} 𝚜𝚛𝚌(x+\prime{x},y+\prime{y})
\end{equation}
	
	
\begin{figure}[H]
  \centering
  \begin{minipage}[b]{0.4\textwidth}
    \includegraphics[width=\textwidth]{dia3.png}
    \caption{\lr{dilation, wSize = (3,3)}}
  \end{minipage}
  \hfill
  \begin{minipage}[b]{0.4\textwidth}
    \includegraphics[width=\textwidth]{dia7.png}
    \caption{\lr{dilation, wSize = (7,7)}}
  \end{minipage}
\end{figure}

\begin{figure}[H]
\center
\includegraphics[scale=0.2]{dia11.png}
\caption{\lr{dilation, wSize = (11,11)}}
\end{figure}

\subsubsection*{\lr{opening}}
در
\lr{opening}
ابتدا 
\lr{dilation}
و سپس
\lr{erosion}
انجام میشود. پس در تصویر داده شده انتظار داریم ابتدا چشم ها سفیدتر شوند و  غبارهاتغعیف شوند و در مرحله بعد مقداری از بدن آدمک که حذف شده بود جبران میشود بنابراین با ضربه کمتری به بدن آدمک میتوان غبارها را حذف کرد . برای دو سایز 7 و ۹ خروجی به صورت زیر است. 
\begin{figure}[H]
  \centering
  \begin{minipage}[b]{0.4\textwidth}
    \includegraphics[width=\textwidth]{open7.png}
    \caption{\lr{opening, wSize = (7,7)}}
  \end{minipage}
  \hfill
  \begin{minipage}[b]{0.4\textwidth}
    \includegraphics[width=\textwidth]{open9.png}
    \caption{\lr{opening, wSize = (9,9)}}
  \end{minipage}
\end{figure}
\subsubsection*{\lr{closing}}
در
\lr{closing}
ابتدا 
\lr{erosion}
و سپس
\lr{dilation}
انجام میشود.ابتدا غبارها تقویت و سپس تضعیف میشوند بنابراین اثر غبار با کرنل همسایز حذف نمیشود. برای دو سایز 7 و ۹ خروجی به صورت زیر است. 
\begin{figure}[H]
  \centering
  \begin{minipage}[b]{0.4\textwidth}
    \includegraphics[width=\textwidth]{clos.png}
    \caption{\lr{closing, wSize = (7,7)}}
  \end{minipage}
  \hfill
  \begin{minipage}[b]{0.4\textwidth}
    \includegraphics[width=\textwidth]{clos9.png}
    \caption{\lr{closing, wSize = (9,9)}}
  \end{minipage}
\end{figure}


\subsection*{سوال هفتم}
در این سوال دو راهکار مد نظر قرار گرفته شد. در هر دو سوال برای اینکه اثرات رنگ را از بین ببریم توسط تابع 
\lr{cv.threshold}  
تصاویر را به باینری تبدل میکنیم. 
\subsubsection*{راه اول}
به صورت حدودی ناحیه را جدا میکنیم و به عنوان الگو به تابع 
\lr{cv.matchTemplate}
میدهیم و نواحی مشابه را میابیم و دور آنها مستطیل میکشیم. این تابع بر اساس معیار کورلیشن نرمال شده عمل میکند و برای هر پیکسل یک احتمال مشابهت ناحیه نزدیک آن مطابق با الگو میدهد و برای انتخاب ناحیه مناسب با مقایسه با ترشولد $0.8$ شکل زیر به دست می آید. محل دقیق با شرط گذاری روی مستطیلها را میتوان محاسبه کرد و با پیدا کردن اولین سیاهی در الگو و اثر شیفت آن میتوان به محل دقیق رسید اما باز چون خودمان اول یک حدس میزنیم ممکن است این راه مطلوب نباشد و بنابراین سوال را دقیق تر توسط روش دوم حل 
میکنیم.

\begin{figure}[H]
\center
\includegraphics[scale=0.2]{m1.png}
\caption{خروجی روش اول}
\end{figure}

\subsubsection*{راه دوم}

در این روش بااستفاده از تابع \lr{cv.findContours
} سطوح تصویر را می یابیم و مستطیل محدود کننده هر ناحیه را محاسبه میکنیم و با شرط گذاشتن روی نسب ضلع های مستطیل به نواحی مطلوب میرسیم و با یک شرط خیلی تقریبی روی محل مستطیل به محل دقیق مستطیل مشخص شده میرسیم که در زیر آورده شده 
است.
\begin{figure}[H]
\center
\includegraphics[scale=0.2]{m2.png}
\caption{خروجی روش دوم}
\end{figure}

\begin{figure}[H]
\center
\includegraphics[scale=0.45]{coords.png}
\caption{مختصات دقیق مستطیل}
\end{figure}

\section{ب)}

\subsection*{۱}
کد مربوط به دستورات خواسته شده نوشته شده است. در صورت فشردن 
\lr{s}
تا زمان توقف ویدیو شروع به ضبط میکند و در نهایت با نام
\lr{saved.avi}
ذخیره میشود. در این سوال از 
\lr{cv.VideoCapture} 
برای خواندن ویدیو از وبکم و 
\lr{VideoWriter}
برای ذخیره ویدیو استفاده میشود.

\subsection*{2}
با استفاده از میانه گیری در طول زمان زمینه عکس به صورت زیر به دست می آید.

\begin{figure}[H]
\center
\includegraphics[scale=0.2]{b2back.png}
\caption{زمینه ویدیو}
\end{figure}
 
سپس با تابع 
\lr{cv.absdiff}
زمینه را از هر فریم کم کرده و با تبدیل قسمتهای سیاه زمینه به سفید نمایش میدهیم. خروجی با نام 
\lr{b2out.avi} 
ذخیره شده است.
\end{document}